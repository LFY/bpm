\documentclass{acmsiggraph}
\usepackage{graphicx}
\usepackage{amsmath}
\usepackage{url}

\usepackage{amsmath, amsfonts}

\DeclareMathOperator{\tr}{tr}
\DeclareMathOperator{\Expect}{E}
\DeclareMathOperator{\KL}{KL}
\DeclareMathOperator{\maximize}{maximize}
\DeclareMathOperator{\subjto}{subject\quad to}
%% The 'graphicx' package allows for the inclusion of EPS figures.

\usepackage{fp}
\usepackage{graphicx}
\usepackage{color}
\definecolor{darkblue}{RGB}{19,46,82}
\definecolor{darkgreen}{RGB}{18,47,30}
\definecolor{stanford}{RGB}{164,0,29}

\definecolor{yypurple}{RGB}{196,66,114}
\definecolor{yygreen}{RGB}{78,155,136}
\definecolor{yyblue}{RGB}{41,41,232}

\usepackage{listings}
\lstset{
language=Python,
basicstyle=\sffamily,
breaklines=true,
tabsize=4,
numbers=left,
numberstyle=\small,
frame=lines,
columns=fullflexible,
showstringspaces=false,
keywordstyle=\bfseries\color{darkblue},
commentstyle=\itshape\color{darkgreen},
stringstyle=\color{stanford}
}

\begin{document}

\title{Program inversion for probabilistic inference}

\author{Lingfeng Yang}
\maketitle

\begin{abstract}

We present a method for efficiently and exactly calculating the probability of
generating a value for a restricted class of nondeterministic functional
programs. The method is based on deriving an execution trace of the program
from the structure of the data and program, which is widely used in program
inversion.  Unlike with program inversion, however, we deal with non-injective
programs and must account all nondeterminism in execution traces producing the
data, because we are calculating probabilities.  Our approach is to rewrite the
original program and data query into a Prolog program expressing the derivation
of the execution trace in a chart parsing-like manner.  The output is an
execution \textit{chart} representing all possible executions that yield the
data, whose structure enables the final probability, which is a nested
sum-product, to be calculated efficently (reusing intermediate parses and
pushing in summations as far as possible).

\end{abstract}

\section{Introduction}

\lstset{language = LISP}
\lstinputlisting[caption='A nondeterministic program generating different structures.']{intro.ss}
\lstinputlisting[caption='What is the probability of generating this data?']{intro-query.ss}

Consider the nondeterministic Scheme program and associated query above.
The program generates trees of varying length.  $f_2$ constructs a tree with two
branches, each of which have subtrees that are the argument to $f_2$.  Consider
$f_1$. $choose$ denotes a nondeterministic evaluation, picking from its
arguments with equal probability.  $f_1$ then recursively constructs a tree of
varying length, evaluating to a node with itself as a subtree, or the number 2.

We would like to compute the probability that this program generates the data
above.  One way to compute the probability is to run the program several times
and tally up the number of times the data was observed versus the total set of
seen data (rejection sampling). However, in this case, we know that the
probability is $1/8 = (1/2)^3$; the only way to generate the data is the
execution of the program where the first alternative of $f_1$ twice
(\verb!`(node ,(f1))!) followed by one evaluation of the second (\verb!2!).
Each such choice is made with probability $1/2$, so the total probability is
$1/8$.

In other words, we can sometimes derive the possible execution traces of the
program producing the data based on the structure of the data and program; it
is not necessary to actually run the program. In general, this helps in
"needle-in-a-haystack" inference problems, where it is difficult to execute the
program and make choices "just right" so that the output matches the data.

We demonstrate this idea concretely in the form of a program transformer that
rewrites a nondeterministic Scheme program and a query to a Prolog program that
derives the execution trees corresponding to the query in a manner very similar
to the CYK algorithm for chart parsing; we scan over substructures of the data
and match them to applications of functions much like one would match
substrings to nonterminals, making sure to reuse the same application if it has
been "parsed" already for the sub-structure of the data in question.

\section{Rewriting functional to logic programs}

Here we give an outline of the conversion process from functional to logic program.
As an example, recall the program from the beginning. We rewrite it to the following Prolog program:

\lstset{language=Prolog}
\lstinputlisting[caption='The first program rewritten as a chart parser for execution traces.']{intro.pl}

Each function in the original program becomes a relation in the Prolog program,
including the return value as one of the arguments, and including extra
arguments for the resulting execution chart. The resulting execution chart is
expressed as a directed acyclic graph, where each node holds a unique
identifier, the name of the function, the arguments, which nondeterministic
choice was taken and how many, and lists of IDs to "successor" nodes that would
correspond to future points in the execution trace:

\lstinputlisting[caption='A sample execution chart.']{intro-out.pl}

Note that this is indeed an execution \textit{chart}, even though it represents
just one possible \textit{trace}; we may account for more than one execution
trace through repeated use of the same subtrace, and multiple references to
subtraces in each of the lists of successors. This is key; it enables the
efficient computation of the final probability.

We now examine the intermediate representations generated to write the above
Prolog program.  

\subsection{Administrative normal form.}

In converting functions to relations, the first issue is one
of nested function calls.  Because Prolog does not support nested calls to
predicates (nor would that make sense), it is necessary to flatten all function
applications in the beginning program, substituting variables where a nested
function application would occur.  As a first step, this is readily
accomplished by conversion to an intermediate representation that looks much
like administrative normal form:

\lstset{language=LISP}
\lstinputlisting[caption='The first program in administrative normal form.']{intro-anf.ss}

Note that we leave choice points alone. They are not seen as values, but more
as control flow constructs in this sense.

\subsection{Sets of unification equations.}

Next, we express the key relationship between "functions" in applicative versus
relational languages; what would be a function returning a value in an
applicative language such as Scheme, becomes a relation where the returned
value is one of the components of the relation; 
i.e., if $f = \lambda x . e$ 
it becomes a relation $f(x, y) :- y = e$. 
This is what allows backward inference; the unification of parts of the data with $x, y$,
and thus the success of the goal $f(x, y)$ and a piece of the execution trace.
We accordingly rewrite each let binding in the administrative normal form to a
unification equation, and later convert equations of the from \verb!(= y (f x))! to \verb!(f x y)! to \verb!f(x, y)!.

\lstinputlisting[caption='Rewriting let bindings into unification goals.']{intro-equations.ss}

\subsection{Unification equations with normalized choices.}

Indeed, it still remains to make the old function symbols truly relations; that
is, the \verb!(= X3 (F1))!  should be \verb!(F1 X3)! (in the Prolog format,
this would be $pF1(X3).$. We save that for the last step. First, we lift out
all random choices to top-level predicates. 

\lstinputlisting[caption='Lifting out all choices to top-level predicates.']{intro-normalized-choices.ss}

This puts all relations in the resulting Prolog program into a normal form; if
it was a deterministic function with no random choices in its body, then it
remains the same. Otherwise, it goes to a Prolog predicate with a disjunction
at the top level, one for each choice.

\subsection{Prolog relations.}

Finally, we convert the old function applications and unification equations
into relations, and add variables representing subtraces, to be built up during
the parsing process. Nondeterministic choice naturally becomes Prolog disjunction.
Also, note the use of \verb!find_at_least_one!, which attempts
to find all solutions to given goal, or fail if there are none. Because of
this, we are indeed performing a tree-based variant of the CYK algorithm, but
specialized for a particular program; we first match "leaf" symbols in the data
in all ways possible, then go one level up and repeat, making sure to reuse
partial parses (which is expressed by \verb!add_if_not_present!, which creates
a new node for the execution trace only if it does not exist already).

\lstinputlisting[caption='Final form before serialization to Prolog format.']{intro-final.ss}

\paragraph{Keeping values of arguments around to be unified.}
There is an issue to address when dealing with original functions that have
arguments, and is the reason behind the duplication of the function body for
\verb!pF2!. It has one argument, $V0$. In finding all ways that \verb!pF2! can
apply to the current sub-structure, we also need to find values for $V0$.
However, this $V0$ also needs to be unified with the actual value in the
original program, which is \verb!(f1)!. But running \verb!find_at_least_one!
does not ground a value to $V0$ that is visible from the outside; only inside
\verb!find_at_least_one! is $V0$ ground to anything. Therefore, we need to run
the body at least once for functions that have arguments. Failure to do will
result in $V0$ being ground to the wrong value, or worse, never being ground,
which can result in infinite loops.

\subsection{Obtaining the probabilities.}

After the execution chart is obtained, the probability of the data may be calculated as a sum-product, recursively defined as

\begin{align*}
P(id, n, \mathbf{s}_1, \ldots \mathbf{s}_m) = \frac{1}{n} \prod^m_{j=1}\sum^{|\mathbf{s}_j|}_{k=1}P(\mathbf{s}_{jk})
\end{align*}

where each $\mathbf{s}_j$ is a vector of DAG-nodes, each of which holds an id
($id$) and number of possible choices $n$ along with its own lists of children
$\mathbf{s}_1, \ldots$. This is efficient if we memoize intermediate results
$P(\mathbf{s}_{jk})$.

\end{document}


